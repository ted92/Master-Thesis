\label{chap:prev_works}
Scalability and analysis on the blockchain
has been taken into consideration
by many researchers in the past years.
This chapter summarizes the most
relevant works on Bitcoin,
blockchain and decentralized cryptocurrencies.
In our previous work\,\cite{Tedeschi:2016:PBB}, we enhance
the importance of paying for having a certain
bandwidth in the Bitcoin network.
Peter~R.~Rizun\,\cite{Rizun:2015:blocksizelimit},
explains how a rational Bitcoin
miner should select transactions from his node mempool,
when creating a new block,
in order to maximize his profit. We discuss that and apply
new data from more recent sources to this idea of
"selecting the right transaction" in a way that a miner could
select the right transactions to earn more money out of the whole
mining process.
Scalability has taken into consideration
in the work of Kyle Croman et al.\,\cite{croman2016}, they
analyze how fundamental bottlenecks in Bitcoin limit the ability
of its current peer-to-peer
overlay network to support substantially higher
throughputs and lower latencies. We are
going to test the throughput as well, comparing it with the one
showed in this paper.
Regarding fees and tolls paid in the Bitcoin blockchain
we refer to the study done in $2014$ from Möser and
Böhme\,\cite{Moser2015}. They analyze the
entire blockchain and make assumptions about
that these "fees" are supposed to substitute miners'
minting rewards in the long run. This paper
contributes empirical evidence from a historical
analysis of agents' revealed behavior concerning
their payment of transaction fees.
Furthermore, to fully understand how is possible
to make money out of the blockchain and mining,
in necessary to have a view of
how \emph{VISA}\,\cite{visa} makes money as well.

\section{Rizun - A Transaction Fee Market Exists Without a Block Size Limit}
\label{sec:rizun}
\subsection{Problems}
A pressing concern exists over the ramifications of changing (or not)
a Bitcoin protocol rule called \emph{block size limit}. This rule sets an
upper bound on the network's transactional capacity, or \emph{throughput}.
The limit is currently set at $1$\,Mb, corresponding roughly to three transactions
per second. When this limit was set, it was over eight hundred times
greater than what was required. However in $2015$, blocks
were filled near capacity and users experienced delays. In $2015$
the transaction rate was over three hundred times larger
than when the block size limit was introduced. One of the
concerns is whether, in the absence of a limit or if the limit
is far above the transactional demand, a healthy transaction
fee market would develop which charges users the full
cost to post transactions. The object
of this paper is to consider whether or not such a fee
market is likely to emerge if miners, rather than the protocol,
limit the block size.
\subsection{Methods}
This paper shows how a Bitcoin miner should
select transactions from his node's mempool
when creating a new block in order to maximize
his profit in the absence of a block size limit.
\emph{Block space supply curve} and
\emph{mempool demand curve} are explained, and the
paper shows how the supply and demand
curves from classical economics are related to the
derivatives of these two curves.
In the paper Rizun claims that the block-size limit determines the
transaction throughput. In this paper he derives
the \emph{miner's profit equation} and then
he introduces two novel concepts called
the \emph{mempool demand curve} and the
\emph{block space supply curve}.
\subsubsection{Miner's Profit Equation}
Every time a block is mined, the miner expects to
generate a revenue $\langle V \rangle$
at hashing cost $\langle C \rangle$ to earn profit per block
\begin{equation}
\label{eq:minerprofit}
\langle \Pi \rangle = \langle V\rangle - \langle C\rangle.
\end{equation}
Miner's profit equation in~\ref{eq:minerprofit}
shows the gain of a miner $\langle \Pi \rangle$,
where the hashing cost is represented as follows:
\begin{equation}
\label{eq:hashingcost}
\langle C\rangle = \eta h\mathcal{T}.
\end{equation}
So the hashing cost $\langle C\rangle$ is
directly dependent from the miner's individual hash rate, $h$,
the cost per hash, $\eta$, and the creation time, $\mathcal{T}$.
Moreover, is important to consider the
expectation value of a miner's revenue per block,
this value is represented with $\langle V\rangle$
and is equal to the amount he would earn if he won
the block multiplied by his probability of
winning. So the expected revenue would be:
$\langle V\rangle = (R + M) h/H$,
where the amount he would earn is the sum of
the block reward, $R$, and the transaction fees, $M$.
His probability of winning, assuming all blocks propagating
instantly, is equal to the ratio of his hash
rate, $h$, to the total hash rate of the Bitcoin network,
$H$. The problem with this equation is that it
does not reflect the miner's diminished chances of
winning if he chooses to publish a block that propagates
slowly to the other miners. If a miner finds first a valid block,
but his solution is received after most miners
are working on another, then his block will likely be discarded.
This effect is called \emph{orphaning}. The
equation, considering the orphaning factor, $\mathbb{P}_{orphan}$, is the following:
\begin{equation}
\label{eq:expectedrevenue}
\langle V\rangle = \left(R + M\right)\frac{h}{H}\left(1 - \mathbb{P}_{orphan}\right).
\end{equation}
Where $P_{orphan}$ increases with the amount of time a block takes
to propagate to other miners. Indeed, if $\tau$ is
the block propagation time, the
probability of orphaning is defined as:
\begin{equation}
\label{eq:orphaning}
\mathbb{P}_{orphan} = 1 - e^{-\frac{\tau}{\mathcal{T}}}.
\end{equation}
In conclusion the \emph{miner's profit equation} is defined as:
\begin{equation}
\label{eq:minerprofiteq}
\langle \Pi \rangle = (R + M)\frac{h}{H} e^{-\frac{\tau}{\mathcal{T}}} -\eta h\mathcal{T}
\end{equation}
A \emph{rational miner} selects which transactions to include in his block in a manner that maximizes
the expectation value of his profit. This selection is explained with the \emph{mempool demand curve}
and the \emph{block space supply curve}.

\subsubsection{The Mempool Demand Curve}
\label{sec:mempooldemand}
The set of transactions that still need to be approved and included
in a block is called \emph{mempool}.
The mempool set is denoted with $\mathcal{N}$ and the number
of transactions contained within it as $n$.
According to the size limit, a block can select a $b \leq n$
transactions from $\mathcal{N}$ to create a
new block $\mathcal{B} \subset \mathcal{N}$.
A block first includes transactions with an higher \emph{fee density}, $\rho$.
This last, is a ratio between
the \emph{transaction fee}, $t_f$ and the \emph{transaction size}, $t_q$.
To construct the mempool demand
curve, is necessary first sorting the mempool from
greatest fee density to least and then
associating an index $\{i: 1,2,\dots,n-1,n\}$ with each
transaction in the resulting list.
The mempool demand curve will be then a graphical
representation of the sum of the fees offered
by each transaction in this sorted list:
\begin{equation}
\label{eq:memdemandcurve}
M_{demand}(b) \equiv \sum_{i=1}^{b} fee_i,
\end{equation}
and the sum of each transaction's size in bytes:
\begin{equation}
\label{eq:transactionsize}
Q(b) \equiv \sum_{i = 1}^{b} size_i.
\end{equation}
The mempool demand curve represents then the maximum fee,
$M_{demand}(b)$
a miner can claim by producing a given quantity $Q(b)$ of blockspace.

\subsubsection{The Block Space Supply Curve}
\label{sec:blockspacesupply}
The size of the block a miner elects to produce controls the fees he attempts to claim, $M(Q)$,
and the propagation time he chooses to risk, $\tau(Q)$. The block space supply curve represents
the fees a miner requires to cover the additional cost of
supplying block space $Q$. This cost grows
exponentially with the propagation time. The equation which
represents this curve is the following:
\begin{equation}
\label{eq:blockspacesupply}
M_{supply}(Q) = R\left(e^{\frac{\Delta \tau (Q)}{\mathcal{T}}} - 1\right),
\end{equation}
where $\Delta \tau (Q) \equiv \tau(Q) - \tau(0)$.
The propagation time ,$\tau$, is just an esteem from
the propagation delay versus the block size.

\subsubsection{Maximizing the Miner's Profit}
To maximize his profit, the miner construct a mempool
demand curve and a space supply curve.
The block size $Q^*$ where the miner's surplus,
$M_{demand} - M_{supply}$, is largest represents
the point of maximum profit. Considering this point $Q^*$ of maximum
profit, Rizun considers three market conditions for Bitcoin transaction
fees: \emph{healthy}, \emph{unhealthy} and \emph{non-existent}.
In a healthy fee market, the miner's surplus is maximized
at a finite quantity of block space, and thus a miner is
incentivized to produce a finite block. In an unhealthy
market, the miner's surplus continually increases with
block space, and therefore a rational miner should produce
an arbitrary large block. In a non-existent market,
including \emph{any} transactions results in a deficit
to the miner, and so the miner is better off
producing an empty block. A rational
miner will produce a big block if his mempool
is full of high fee density transactions, and
will produce an empty block if no transactions pay a fee sufficient
to offset the orphaning risk.

\subsection{Results}
%TODO: Conclusions from this paper
In conclusion, they show that a transaction fee market should
emerge without a block size limit if miners
include transactions in a manner that maximizes
the expectation value of their profit. A
critical step in establishing this result was their
calculation of the miner’s cost to supply
additional block space by accounting for orphaning risk.

\section{Möser \& Böhme - Trends, Tips, Tolls: A Longitudinal
study of Bitcoin Transaction Fees}
\label{sec:moser}
\subsection{Problems}
The Bitcoin protocol supports optional direct payments
from transaction partners to miners, also called \emph{fees}.
Acknowledging their
role for the stability of the system, the right level of
transaction fees is a hot topic of normative debate. The actual
costs of the system are not extensively studied yet. Disregarding
intangible factors of (in)convenience, Bitcoin may not be as cheap for
consumers as it appears. The main problems/questions
that this paper focuses on are:
\begin{enumerate}[noitemsep]
	\item Do higher transaction fees lead to faster confirmation?
	\item Do impatient users offer higher fees?
	\item Do mining pools enforce strictly positive fee systematically
	(excluding zero-fee transactions)?
\end{enumerate}

\subsection{Methods}
They enhance the definition of transaction fee, which is encoded
as difference between the sum of all inputs and the sum of all outputs
of a transaction. Then to study trends of Bitcoin transaction fee
conventions over the past couple of years, they combine data from
different sources. They load the blockchain by parsing
the block files of the Bitcoin Core client\,\cite{bitcoincore}
and extract information on the size of the block and transactions.
Additional data is fetched from \url{blockchain.info}, such as
information about miners. Furthermore, data on bitcoin exchange rate
is taken from \url{coindesk.com}, which provides an average Bitcoin
price in \gls{usd}. The time range selected for the analysis
is in between January\,$2011$ and August\,$2014$. To answer
question $(1)$, they compared time when a transaction is first
seen on the network and the timestamp of the block that includes
the transaction, calculating in that way transaction latency, $t_l$.
They analyze a representative subset of $9000$ transactions
randomly chosen from all eligible transaction between June
$2012$ and May $2013$, then to answer question $(2)$ they
compute for each transaction the holding time, which is the period
until the output was spent again, and compared their fees to see if
they are higher. To answer question $(3)$ is necessary get
information about major mining pools. They used data
from \url{blockchain.info} to retrieve useful information about miners
and major miners were analyzed such as \emph{AntPool}, \emph{50BTC},
\emph{BitMinter}, \emph{Slush}, \emph{ASICMiner} and more.
\subsection{Results}
\subsubsection{Trends}
Overall, they claim that Bitcoin transaction fees are lower than $0.1\%$ of
the transmitted value, which is significant below the fees charged
by conventiaonal payment systems. It appeared to them that hard size limit
do not (yet) significantly drive the level of transaction fees.
In our thesis we want to test if this is still true.
Regarding trends for the fees paid per transaction over time,
the first notable change from $0$ and $0.01$\,\bitcoin~fee
occurs after June $2011$, transactions
with fee of $0.0005$\,\bitcoin~appear and account for
about $20$-$30\%$ of all transaction. In the second quarter
of $2012$ the transactions paying $0.0005$\,\bitcoin~raised
to $60$-$70\%$ of all transactions. In the fourth quarter of
$2012$, $30$-$40\%$ of all transactions were paying a
fee of $0.001$\,\bitcoin. In May $2013$, the nominal value
of $0.001$\,\bitcoin~makes space for a tenth: $0.0001$\,\bitcoin.
This fee level stays on and gains a share of more than $70\%$
towards $2015$. In order to reason about these changes, they
mapped important events in the Bitcoin ecosystem. Generally,
there seem to be two main reasons for shift in trends:
changes to the Bitcoin reference implementation and actions by large
intermediaries in the ecosystem. The emergence of $0.0005$\,\bitcoin~fees
in June $2011$ can be mapped to the release of version $0.3.23$
of the Bitcoin Core client, which reduced the default transaction fee
from $0.01$\,\bitcoin~to $0.0005$\,\bitcoin. The raise of these
last transaction fees in the second quarter of $2012$ is probably
due to the launch of the gambling website \emph{SatoshiDice}\,\cite{satoshidice}.
On May $2013$, version $0.8.2$ of Bitcoin Core was released.

\subsubsection{Tips}
There is a small share of transactions that did not offer fee to miners,
most of them offered default fee amount but some of them
were even willing to pay a higher fee. A plausible reason
is that paying more in fee leads to a faster confirmation.
After the analysis turned out that half of all zero-fee transactions
had to wait more than $20$ minuets for their first confirmation.
In contrast to that, paying a $0.0005$\,\bitcoin~fee lead to an
inclusion into a block in half of the time. $10\%$ of all zero-fee
transactions took almost $4$\,hours to confirm, in contrast
to $40$\,minutes for transactions paying a $0.0005$\,\bitcoin~
fee. The difference between paying $0.0005$\,\bitcoin~or
$0.001$\,\bitcoin~fee is not as pronounced, but the difference
in medians are still statistically and economically significant.

\subsubsection{Tolls}
Analysis on pool behavior regarding a possible systematic
exclusion of zero-fee transactions has been done.
Shares have shifted between pools quite extensively.
In $2013$, BTC Guild had a market share of up to $40\%$,
in $2014$ both GHash.IO and Discus Fish ousted this pool.
Also, the share of other pools has risen in $2014$. Previous
incumbents like Slush or $50$BTC have lost popularity.
Possible reasons include economic and technical factors,
like pool fees, service availability, or robustness against
attacks. Given the dominance of a few mining pools, they
evaluated whether some pools systematically enforce fees.
The results show that two pools, Discus Fish and Eligius,
have a considerably higher share of blocks without any
zero-fee transaction, with $30.6\%$ for Eligius and $62.5\%$
for Discus Fish, in contrast to an average of $14.4\%$.
Over than that though, there is no clear evidence for
enforcement of strictly positive transactions fees.
 
\section{Croman - On Scaling Decentralized Blockchains}
\label{sec:croman}
\subsection{Problems}
The increasing popularity of blockchain-based cryptocurrencies
has made scalability a primary and urgent concern. The main question
that this paper focuses on is the following:
\begin{quote}
\emph{Can decentralized blockchains be scaled up to match the performance
of a mainstream payment processor? What does it take
to get there?}
\end{quote}
At the time of writing, the Bitcoin blockchain took $10$\,min or longer
to confirm transactions, achieving $7$\,transactions/sec maximum
throughput. Visa credit card confirms a transaction within seconds
and processes $2000$\,transactions/sec ona verage with peaks
of $56,000$\,transactions/sec. This paper aims to place
exploration of blockchain scalability on a scientific footing.
Bitcoin community has put forth various proposals to modify
the key systems parameters of block size and block interval.
In this paper they show that such scaling by reparametrization
can achieve only limited benefits. This because because Bitcoin
generates a lot of network traffic, due to its
decentralization. There are a lot of peers in the network
and they all have to interact. To ensure that most
of the nodes in the overlay network have sufficient
throughput they set two guidelines:
\begin{itemize}[noitemsep]
	\item \textbf{Throughput limit.} The block size should not exceed $4$\,MB
	given $10$\,minutes average block interval. Corresponding at maximum
	$27$\,transactions/sec.
	\item \textbf{Latency limit.} The block interval should not be smaller
	than $12$\,seconds.
\end{itemize}
The community also proposed radically different scaling approaches,
and introduced mechanisms such as Corallo's relay network,
a centralized block propagation mechanism.
One of the main contribution of this paper was to \emph{quantify}
Bitcoin's current scalability limits within its decentralized components.
Their findings leaded them to the position that \emph{fundamental protocol
redesign is needed for blockchains to scale significantly while retaining their
decentralization}. Plus, scalability is not a single metric and
measurement and understanding of many important metrics,
like \emph{fairness} or \emph{mining power utilization}, are lacking.
Monitoring and measuring a decentralized blockchain from only
a few vantage points poses significant challenges. In this paper
they call for better measurements techniques, by continuously monitor
the health of the decentralized system to answer key questions such as:
\emph{"To what extent can we push system paramteres without sacrificing security?"}.

\subsection{Methods}
In this paper they manly focused on:
\begin{itemize}[noitemsep]
	\item \textbf{Maximum throughput.} At the time of writing ($2016$) maximum
	throughput was $3$-$7$\,transactions/sec. Number constrained by
	$Q$ and $\mathcal{T}$ .
	\item \textbf{Latency.} Time for a transaction to confirm, $t_l$. A transaction
	is considered confirmed when it is included in a block, roughly $10$\,minutes
	expectation.
	\item \textbf{Bootstrap time.} The time it takes to a new node to download and
	process the history necessary to validate the current system
	state. In $2016$ that was roughly $4$\,days. 
	\item \textbf{\gls{cpct}.} The cost in \gls{usd} of resources consumed
	by the entire Bitcoin system to confirm a single transaction. It could be
	summarized in:
	\begin{enumerate}
		\item \emph{Mining:} Expended by miners generating the proof of work
		for each block.
		\item \emph{Transaction validation:} The cost of computation necessary
		to validate that a transaction can spend the outputs referenced by its inputs,
		dominated by cryptographic verifications.
		\item \emph{Bandwidth:} The cost of network resources required to receive
		and transmit transactions, blocks and metadata.
		\item \emph{Storage:} The cost of storing all currently spendable
		transactions, which is necessary for miners and full nodes to perform
		transaction validation, and of storing the blockchain's historical data,
		which is necessary to bootstrab new node that join the network.
	\end{enumerate}
\end{itemize}
The cost per transaction for Bitcoin was calculated performing a
back-of-the-envelope calculation by summing up the electricity
consumed by the network as a whole, as well as the hardware cost
of mining equipment. They projected their estimates based on the
\emph{AntMiner S5+} mining hardware\,\cite{mining_hw}. They assume
a $1$\,year effective lifetime for the hardware and that the average
hashing rate of the network is $450,000,000$\,GH/s. Furthermore,
they assume an average price per KWh of $0.1\$$. Two scenarios are
possible, the first is when the Bitcoin network is operating at
maximum throughput of $3$-$7$\,transactions/sec. This
limit is constrained by the $1$\,MB block size limit and the
variable transactions size. The lower bound is inferred
from the average transaction size of $500$\,bytes, while the
upper bound is based on an unusually small transactions
size of $250$\,bytes. The second scenario is based on the
average throughput of Bitcoin network, which is based on
statistics collected in October $2015$, and it resulted to be
of $1.57$\,transactions/sec. They show then a Bitcoin cost
breakdown assuming that the entire network contains
$5400$ full nodes and they evince that is a fallacy to assume
that transaction costs necessarily have to be offset by
transaction fees. Indeed, the costs of running full nodes
may be offset by financial externalities such as selling items
whose costs computational time for a node or confirming a
transaction without trusting third parties. Said so, is important
to enhance that miners are bereft of these two factors and
they need to be compensated. According to new measurements
on the block propagation time, they also defined \textbf{X\% effective
throughput} as follows:
\[
\text{X\% effective throughput} = Q / (\text{X\% block propagation delay})
\]
Considering that the propagation delay is, for $1$\,MB block size and
$\text{X}\% = 90\%$ of block propagation, $2.4$\,minutes, they calculated
an X\% effective throughput of $55$\,Kbps $\equiv$ $26$\,tx/sec
having  $\text{X}\% = 90\%$ of block propagation.
\subsubsection{Throughput limit}
They observed that the block size $Q$, and interval $\mathcal{T}$,
must satisfy:
\[
\frac{Q}{\text{X\% effective throughput}} < \mathcal{T}
\]
having in that way, for a $10$\,minutes block interval a block size
that should not exceed $4$\,MB for $\text{X}=90\%$ and $38$\,MB for
$\text{X}=50\%$. Given $\mathcal{T} = 10$\,minutes the block size
should not exceed $4$\,MB, corresponding to a throughput
of at most $27$\,transactions/sec.
\subsubsection{Latency limit}
To improve the system's latency it could be enough
to reduce the block interval. To maintain effective
throughput that would also require a reduction in the
block size. Propagating a block smaller than $80$\,KB would
not make full use of the network's bandwidth, as latency would
still be a significant factor in the block's propagation time.
To propagate a $80$\,KB block to $90\%$ of the nodes would
take roughly $12$\,seconds. In conclusion, to retain at least $90\%$
effective throughput and fully utilize the bandwidth of the network,
the block interval should not be smaller than $12$\,seconds.
\subsection{Results}
More difficult to measure metrics could also reveal scaling
limitations, example \emph{fairness}. Their measurement results
suggest that top $10$\,\% nodes receive a $1$\,MB block $2.4$\,minutes
earlier than the bottom $10$\,\%, meaning that some miners could
obtain a significant lead over others solving hash puzzles. In the end,
they want to rethink the design of a scalable blockchain, organizing
it around a decomposition of the Bitcoin system into a set of abstraction
layers that they called \emph{planes}. In a hierarchical of
dependency from bottom to the top the layers are:
\begin{enumerate}[noitemsep]
	\item \textbf{Network Plane.} It propagates transaction messages. Bitcoin's
	network protocol do not fully utilize underlying network bandwidth,
	making Bitcoin's Network Plane the bottleneck in transaction processing.
	A solution could be to avoid denial-of-service by propagation
	of invalid transactions, a node must fully receive and validate a transaction
	before further propagations.
	\item \textbf{Consensus Plane.} Functionality that mines
	blocks and reaches consensus on their integration
	in the blockchain. It receives
	messages from Network Plane and outputs transactions ready to be insert
	in the system ledger. Bitcoin's blockchain protocol has a three-way-tradeoff
	between, \emph{consensus speed}, \emph{bandwidth} and \emph{security}, and
	the increment of two of them leads to the loss of the third. For example,
	if the first two are improved then there is loss in the mining power
	that secures the system.
	\item \textbf{Storage Plane.} It functions as a global memory
	that stores and provides availability for authenticated data produced
	by the Consensus Plane. Storage Plane in Bitocin only supports
	\texttt{writes} operations that append data and doesn't support
	\emph{delete} operations. The only supported \emph{read} operation
	downloads the entire ledger, a process that require four days. The
	community has proposed ideas such as \gls{utxo} data structure.
	\item \textbf{View Plane.} A view is a data structure derived from the
	full ledger whose state is obtained by applying all transactions. For
	Bitcoin miners, it is unnecessary to operate on the full ledger that
	stores the entire transaction history. Miners and nodes in Bitcoin
	locally compute and operate on a view of the ledger called \gls{utxo} set,
	which specifies the current balance of all entities in the system.
	Bitcoin requires all consensus nodes to verify all transactions, and
	based on the result of the computation, every node needs to update its
	view, e.g. \gls{utxo} sets, locally and generating then the same conclusion
	of all other nodes in the system, representing in that way
	an honest set of consensus nodes, keeping an high availability.
	\item \textbf{Side Plane.} Allows off-the-main-chain consensus.
\end{enumerate}

\section{Assumptions} %TODO: improve assumptions (maybe)
\label{sec:assumptions}
The main problems that we should focus on
are related to blockchain scalability, performance and
see whether the constant increasing amount of transactions
per day affects the way miners include transactions
or the way clients offer fees to miners.
We assume then, having information from the
previous listed papers, that analyzing the last $3$\,years
of transactions might produce interesting
results about how this (not anymore) new but still
cryptic system is evolving and changing its
properties according to achieve better performance
even after it scaled drastically in the last couple of
years
and still be able to compensate miners and users.